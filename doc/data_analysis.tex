\documentclass[modern,pdftex]{aastex631}
\def\lya{\mbox {Ly$\alpha$}}
\def\lyb{\mbox {Ly$\beta$}}

\begin{document}

{\bf Excerpt from COS / PG quasars paper on UV doublet analysis:}

We characterize the doublet absorption features (N~V
$\lambda\lambda$1238, 1243; O~VI $\lambda\lambda$1028, 1032; and P~V
$\lambda\lambda$1119, 1128) in the quasar spectra using model
fits. Our goal is to measure the kinematics of the outflowing gas
associated with these features (mass, momentum, and energy estimates
are beyond the scope of the present paper, except for a few special
cases discussed in Section X. If the lines within these doublets were
unblended, fits to the intensity profiles of the individual lines
would thus be sufficient. However, the doublet lines are often
strongly blended because of (1) strong blueshifts due to high outflow
velocities and (2) broad line profiles due to multiple clouds along
the line of sight and/or large linewidths. We thus adopt the doublet
fitting procedure of Rupke et al. (2005), which is optimized for
blended doublets. In this method, the total absorption profiles of a
feature are fit as the product of multiple doublet components. Each
component is a Gaussian in optical depth $\tau$ vs. wavelength with a
constant covering factor $C_f$. Within each doublet the two lines have
a constant $\tau$ ratio. This allows us to simultaneously fit $\tau$
and $C_f$, which are otherwise degenerate in the fit of a single
line. The free parameters in the fit to each doublet component are
thus $C_f$, peak $\tau$, velocity width, and central wavelength.

The general expression for the normalized intensity of a doublet
component is
\begin{eqnarray}
  I(\lambda) = 1 - C_f + C_f e^{-\tau_\mathrm{low}(\lambda)-\tau_\mathrm{high}(\lambda)},
\label{eq:I_lambda}
\end{eqnarray}
where $C_f$ is the line-of-sight covering factor (or the fraction of
the background source producing the continuum that is covered by the
absorbing gas; though scattering into the line of sight can also play
a role) and $\tau_\mathrm{low}$ and $\tau_\mathrm{high}$ are the
optical depths of the lower- and higher-wavelength lines in the
doublet (Rupke et al. 2005). The covering factor is the same for both
lines of the doublet. The peak (and total) optical depths of the
resonant doublet lines in O~VI, N~V, and P~V are related by a constant
factor $\tau_{low}/\tau_{high} = 2.00$ because of the 4-fold
degeneracy in the upper state of the higher energy transition compared
to the 2-fold degeneracy in the lower state. (The higher degeneracy is
due in turn to its higher total angular momentum quantum number
$j$). For more than one doublet component, we use the product of the
intensities of the individual components, which is the
partially-overlapping case of Rupke et al. (2005).

Because the doublet profile shape--i.e., relative depths of the two
lines and trough shape--does not change significantly above optical
depths $\tau_{high}$ of a few, we set a limit of $\tau_{high} \le
5$. Out of 59 O~VI components, 19 have $\tau_{high} = 5$, or 32\%. For
N~V, 13 of 62 components have $\tau_{high} = 5$, or 21\%.

The results from these fits are also used to calculate the total
velocity-integrated equivalent widths of the absorbers in the object's
rest frame,
\begin{eqnarray}
  W_{\rm eq} = \int [1 - f(v)] dv, 
\label{eq:weq}
\end{eqnarray}
the weighted average outflow velocity, 
\begin{eqnarray}
  v_{\rm wtavg} = \frac{\int v [1 - f(v)]dv }{W_{\rm eq}}, 
\label{eq:vwtavg}
\end{eqnarray}
and the weighted outflow velocity dispersion,
\begin{eqnarray}
  \sigma_{\rm wtavg} = \left(\frac{\int (v - v_{\rm wtavg})^2 [1 -
  f(v)] dv}{W_{\rm eq}}\right)^{\frac{1}{2}},
\label{eq:sigmawtavg}
\end{eqnarray}
a measure of the second moment in velocity space of the absorbers in
each quasar. These quantities are similar to those defined by Trump et
al.\ (2006), but without the constraints on depth, width, or
velocity. These constraints have little effect on the results for our
sample, but we find it useful to include possibly inflowing absorbers.

The optical depths and covering factors derived from our fitting
scheme are approximations. Though it is a physically-motivated way to
decompose strongly-blended doublets, the method implicitly assumes
that the velocity dependences of $C_f$ and $\tau$ can be described as
the sum of discrete independent Gaussians. In reality, they are
probably more complex functions of velocity (e.g., Arav et al. 2005,
2008). In several cases--the N~V absorbers in PG~1001$+$054,
PG~1411$+$442, PG~1617$+$175, and PG2214$+$139, and the O~VI absorbers
in PG~1001$+$054, PG~1004$+$130)--the fits include very broad
components that cannot be distinguished from complexes of narrower
lines given the data quality. In two O~VI absorbers (PG~0923$+$201 and
PG~1309$+$355), there is no data on the blue line because it is
contaminated by geocoronal \lya, so any constraints on $\tau$ and
$C_f$ come solely from line shape. Finally, in four O~VI fits
(PG~1001$+$054, PG~1004$+$130, PG~1126$-$041, and PG~1617$+$175, the
\lyb\ and O~VI absorption blend together and cannot be easily
separated in the fit. In three of these cases (all but PG~1001$+$054),
we simply fit the visible absorption as due solely to O~VI at
wavelengths in which there is at least some O~VI absorption
contributing to the spectrum. For the fourth case, we are able to
roughly separate the lines by fitting only down to a specific
wavelength.

\end{document}
